
%%%%%%%%%%%%%%%%%%%%%%%%%%%%%%%%%%%%%%%%%%%%%%%%%%%%%%%%%%
%%
%% This is the "PREAMBLE". Here we define the type of document and load in any packages we might want. You can also set parameters %% % and create your own short-hand here.
%%
%%%%%%%%%%%%%%%%%%%%%%%%%%%%%%%%%%%%%%%%%%%%%%%%%%%%%%%%%%

 \documentclass[9pt]{article}
 
 \def\solutions{1}


 \usepackage{amsmath}
 \usepackage{amssymb}
 \usepackage{graphicx}    % needed for including graphics e.g. EPS, PS \usepackage{tikz}
 \usepackage{tikz}
 \usepackage{tikzsymbols}
 \usepackage{relsize}
 \usetikzlibrary{patterns,decorations.pathreplacing,shapes,arrows}
 \usepackage{algorithm2e}
 \topmargin -2.5cm        % read Lamport p.163
 \oddsidemargin -0.04cm   % read Lamport p.163
 \evensidemargin -0.04cm  % same as oddsidemargin but for left-hand pages
 \textwidth 16.59cm
 \textheight 25.94cm
% \pagestyle{empty}        % Uncomment if don't want page numbers
 \pagenumbering{gobble}
 \parskip 7.2pt           % sets spacing between paragraphs
 %\renewcommand{\baselinestretch}{1.5} 	% Uncomment for 1.5 spacing between lines
 \parindent 0pt		  % sets leading space for paragraphs
 \usepackage{multicol}
 \setlength{\columnsep}{1cm}
% No date in header
\date{}

\usepackage{hyperref}
\hypersetup{
    colorlinks=true,
    linkcolor=blue,
    filecolor=magenta,      
    urlcolor=cyan,
}
\usepackage{amsthm}
\usepackage{fancyhdr}
\pagestyle{fancy}
\setlength{\headsep}{36pt}

\usepackage{hyperref}

\newcommand{\lp}{\left(}
\newcommand{\rp}{\right)}
\newcommand{\lb}{\left[}
\newcommand{\rb}{\right]}
\newcommand{\ls}{\left\{}
\newcommand{\rs}{\right\}}
\newcommand{\lbar}{\left|}
\newcommand{\rbar}{\right|}
\newcommand{\ld}{\left.}
\newcommand{\rd}{\right.}

\newcommand{\myexists}{\exists \hspace{.3mm}}

\newcommand{\hs}{\hspace{.75mm}}
\newcommand{\bs}{\hspace{-.75mm}}
\newcommand{\nin}{\noindent}

\newcommand{\fx}{f\bs\left( x \right)}
\newcommand{\gx}{g\bs\left( x \right)}
\newcommand{\qx}{q\bs\left( x \right)}

\newcommand{\nn}{\nonumber}

\newcommand{\vfive}{\vspace{5mm}}
\newcommand{\vthree}{\vspace{3mm}}

\newcommand{\fof}[1]{f\lp #1\rp}
\newcommand{\gof}[1]{g\lp #1\rp}
\newcommand{\qof}[1]{q\lp #1\rp}

\newcommand{\myp}[1]{\left( #1 \right)}
\newcommand{\myb}[1]{\left[ #1 \right]}
\newcommand{\mys}[1]{\left\{ #1 \right\}}
\newcommand{\myab}[1]{\left| #1 \right|}

\newcommand{\myj}{_j}
\newcommand{\myjp}{_{j+1}}
\newcommand{\myjm}{_{j-1}}

\newcommand{\f}[1]{f\hspace{-1mm}\left( #1 \right)}
\newcommand{\fp}[1]{f'\hspace{-1mm}\left( #1 \right)}
\newcommand{\g}[1]{g\hspace{-1mm}\left( #1 \right)}
\newcommand{\gp}[1]{g'\hspace{-1mm}\left( #1 \right)}
\newcommand{\q}[1]{q\hspace{-1mm}\left( #1 \right)}
\newcommand{\qp}[1]{q'\hspace{-1mm}\left( #1 \right)}
\newcommand{\Px}[1]{P\hspace{-1mm}\left( x_{#1} \right)}
\newcommand{\Qx}[1]{Q\hspace{-1mm}\left( x_{#1} \right)}

\newcommand{\tten}[1]{\times 10^{#1}}

\newcommand{\aij}[1]{a_{#1}}
\newcommand{\bij}[1]{b_{#1}}
\newcommand{\rij}[1]{r_{#1}}

\newcommand{\R}[1]{\mathbb{R}^{#1}}

\newcommand{\ith}{i^{\textrm{th}}}
\newcommand{\jth}{i^{\textrm{th}}}
\newcommand{\kth}{i^{\textrm{th}}}

\newcommand{\inv}[1]{{#1}^{-1}}

\newcommand{\bx}{\mathbf{x}}
\newcommand{\bv}{\mathbf{v}}
\newcommand{\bw}{\mathbf{w}}
\newcommand{\by}{\mathbf{y}}
\newcommand{\bb}{\mathbf{b}}
\newcommand{\be}{\mathbf{e}}
\newcommand{\br}{\mathbf{r}}
\newcommand{\xhat}{\hat{\mathbf{x}}}

\newcommand{\beq}{\begin{eqnarray}}
\newcommand{\eeq}{\end{eqnarray}}

\newcommand{\ben}{\begin{enumerate}}
\newcommand{\een}{\end{enumerate}}

\newcommand{\bsq}{\mathsmaller{\blacksquare}}

\newcommand{\iter}[1]{^{\myp{#1}}}

% matrix macro
\newcommand{\mymat}[1]{
\left[
\begin{array}{rrrrrrrrrrrrrrrrrrrrrrrrrrrrrrrrrrrrrrr}
#1
\end{array}
\right]
}

\newcommand{\makenonemptybox}[2]{%
%\par\nobreak\vspace{\ht\strutbox}\noindent
\item[]
\fbox{% added -2\fboxrule to specified width to avoid overfull hboxes
% and removed the -2\fboxsep from height specification (image not updated)
% because in MWE 2cm is should be height of contents excluding sep and frame
\parbox[c][#1][t]{\dimexpr\linewidth-2\fboxsep-2\fboxrule}{
  \hrule width \hsize height 0pt
  #2
 }%
}%
\par\vspace{\ht\strutbox}
}
\makeatother

\newcommand{\smallaug}[1]{
\left[
\begin{array}{rr|r}
#1
\end{array}
\right]
}

%%%%%%%%%%%%%%%%%%%%%%%%%%%%%%%%%%%%%%%%%%%%%%%%%%%%%%%%%%
%%
%% End of PREAMBLE
%%
%%%%%%%%%%%%%%%%%%%%%%%%%%%%%%%%%%%%%%%%%%%%%%%%%%%%%%%%%%



% ======================================================================================
% Actual document starts here. 
% PLEASE FILL IN YOUR NAME AND STUDENT ID.
% ======================================================================================
\begin{document}

\lhead{{\bf CSCI 3104, Algorithms \\ Problem Set 2 (50 points)} }
\rhead{Name: \fbox{Felipe Lima} \\ ID: \fbox{109290055} \\ {\bf Due January 29, 2021 \\ Spring 2021, CU-Boulder}}
\renewcommand{\headrulewidth}{0.5pt}

\phantom{Test}

\begin{small}
\textit{Advice 1}:\ For every problem in this class, you must justify your answer:\ show how you arrived at it and why it is correct. If there are assumptions you need to make along the way, state those clearly.
\vspace{-3mm} 

\textit{Advice 2}:\ Verbal reasoning is typically insufficient for full credit. Instead, write a logical argument, in the style of a mathematical proof.\\
\vspace{-3mm} 

\textbf{Instructions for submitting your solution}:
\vspace{-5mm} 

\begin{itemize}
	\item The solutions \textbf{should be typed} and we cannot accept hand-written solutions. \href{http://ece.uprm.edu/~caceros/latex/introduction.pdf}{Here's a short intro to Latex.}
	\item You should submit your work through \href{https://www.gradescope.com/courses/218966}{\textbf{Gradescope}} only.
	\item The easiest way to access Gradescope is through our Canvas page. There is a Gradescope button in the left menu.
	\item Gradescope will only accept \textbf{.pdf} files.
	\item \href{https://www.youtube.com/watch?v=u-pK4GzpId0&feature=emb_logo}{It is vital that you match each problem part with your work.} Skip to 1:40 to just see the matching info.
\end{itemize}
\vspace{-4mm} 
\end{small}

\hrulefill
\pagebreak



\ben
%%%%%%%%%%%%%%%%%%%%%%%%%%%%%%%%%%%%%%%%%%%%%%%%%%%%%%%%
% PROBLEM  ONE %% PROBLEM  ONE %% PROBLEM  ONE %% PROBLEM  ONE %% PROBLEM  ONE %
%==============================================================================
% Problem 1: Logarithms & Exponents Review
%==============================================================================
% PROBLEM  ONE %% PROBLEM  ONE %% PROBLEM  ONE %% PROBLEM  ONE %% PROBLEM  ONE %
%%%%%%%%%%%%%%%%%%%%%%%%%%%%%%%%%%%%%%%%%%%%%%%%%%%%%%%%

\item The following problems are a review of logarithm and exponent topics.

\begin{enumerate}

\item Solve for $x$.

	\begin{enumerate}
	\item $3^{2x} =  81$
	\item $3(5^{x-1}) = 375 $
	\item $ \log_3 x^2  = 4  $
	\end{enumerate}


\item Solve for $x$.

	\begin{enumerate}
	\item $x^2 - x = \log_5 25 $
	\item $ \log_{10} (x+3) - \log_{10} x = 1 $
	\end{enumerate}

\item Answer each of the following with a TRUE or FALSE. 

	\begin{enumerate}
	\item $ a^{\log_a x} = x $
	\item $ a^{\log_b x} = x $
	\item $ a = b^{\log_b a} $
	\item $\log_a x = \frac{\log_b x}{\log_b a} $
	\item $\log b^m = m \log b $
	\end{enumerate}
	
\end{enumerate}

  \if\solutions1
  \vspace{2mm}
  
  \textbf{Solution:}   \\
%==============================================================================
% STUDENTS: TYPE YOUR SOLUTIONS HERE. (Between \textbf{Solution:} and \fi )
%==============================================================================

\begin{enumerate}

	\item Solve for $x$.

	\begin{multicols}{3}			
		\begin{enumerate}
			\item $3^{2x} =  81$ \\
			$3^{2x} =  3^4$ \\ 
			$2x = 4$ \\
			$x = 2$
			\item $3(5^{x-1}) = 375 $ \\ 
			$ 5^{x-1} = 125 $ \\ 
			$5^{x-1} = 5^{3} $ \\ 
			$x-1 = 3$ \\ 
			$x = 4$
			\item $ \log_3 x^2  = 4  $ \\
			$ 3^4  = x^2 $ \\ 
			$x=\pm \sqrt{3^4}$ \\
			$x=\pm 9$
		\end{enumerate}	
	\end{multicols}

	\item Solve for $x$
	\begin{multicols}{2}			
		\begin{enumerate}
			\item $x^2 - x = \log_5 25 $\\
			$x^2 - x = \log_5 (5^2)$\\
			$x^2 - x = 2\log_5 5$\\
			$x^2 - x = 2$\\
			$x^2 - x -2 = 0$\\
			$(x-2)(x+1)$\\
			$x=2,x=-1$

			\item $ \log_{10} (x+3) - \log_{10} x = 1 $\\
			$\log_{10}(x+3)=1+\log_{10}(x)$\\
			$x+3 = 10$\\
			$9x = 3$\\
			$x = \frac{1}{3}$\\
		\end{enumerate}	
	\end{multicols}

	\item TRUE or FALSE
	\begin{multicols}{5}			
		\begin{enumerate}
			\item TRUE
			\item FALSE
			\item TRUE
			\item TRUE
			\item TRUE
		\end{enumerate}	
	\end{multicols}
	
\end{enumerate}


\fi

\newpage


%%%%%%%%%%%%%%%%%%%%%%%%%%%%%%%%%%%%%%%%%%%%%%%%%%%%%%%%
% PROBLEM TWO %% PROBLEM TWO %% PROBLEM TWO %% PROBLEM TWO %% PROBLEM TWO %
%==============================================================================
% Problem 2: Limits at Infinity Review
%==============================================================================
% PROBLEM TWO %% PROBLEM TWO %% PROBLEM TWO %% PROBLEM TWO %% PROBLEM TWO %
%%%%%%%%%%%%%%%%%%%%%%%%%%%%%%%%%%%%%%%%%%%%%%%%%%%%%%%%


\vspace{5mm}

\item Compute the following limits at infinity. Show all work and justify your answer.

\begin{enumerate}
\item $ \displaystyle \lim_{x \to \infty} \frac{3x^3 + 2}{9x^3 - 2x^2 +7} $
\item $ \displaystyle \lim_{x \to \infty} \frac{x^3}{e^{x/2}} $
\item $ \displaystyle \lim_{x \to \infty} \frac{\ln x^4}{x^3} $
\end{enumerate}

\if\solutions1
\vspace{2mm}

\textbf{Solution:} \\
%==============================================================================
% STUDENTS: TYPE YOUR SOLUTIONS HERE. (Between \textbf{Solution:} and \fi )
%==============================================================================
			
\begin{enumerate}
	\begin{multicols}{2}
		\item $ \displaystyle \lim_{x \to \infty} \frac{3x^3 + 2}{9x^3 - 2x^2 +7} $
		
		$ \displaystyle = \lim_{x \to \infty} \frac{3x^3 + 2}{9x^3 - 2x^2 +7}$

		$\displaystyle = \lim_{x \to \infty}\frac{3+\frac{2}{x^3}}{9-\frac{2}{x^2}+\frac{7}{x^3}}$

		$\displaystyle = \frac{3}{9}$

		$ \displaystyle = \frac{1}{3}$\\
		\vskip 1in

		\item $ \displaystyle \lim_{x \to \infty} \frac{x^3}{e^{x/2}} $\\
		$ \displaystyle =\lim_{x \to \infty} \frac{3x^2}{e^{x/2}\frac{1}{2}} $ L'hopital\\
		$ \displaystyle =\lim_{x \to \infty} \frac{6x^2}{e^{x/2}} $\\
		$ \displaystyle =\lim_{x \to \infty} \frac{12x}{e^{x/2}\frac{1}{2}} $ L'hopital \\ 
		$ \displaystyle =\lim_{x \to \infty} \frac{24x}{e^{x/2}} $\\
		$ \displaystyle =\lim_{x \to \infty} \frac{24}{e^{x/2}\frac{1}{2}} $ L'hopital \\ 
		$ \displaystyle =\lim_{x \to \infty} \frac{48}{e^{x/2}} $\\
		$ \displaystyle =\frac{48}{\infty}$

		$ = 0 $
	\end{multicols}

	\item $ \displaystyle \lim_{x \to \infty} \frac{\ln x^4}{x^3} $
	
	$\displaystyle = \lim_{x \to \infty} \frac{\frac{4}{x}}{3x^2}$ L'hopital

	$\displaystyle = \lim_{x \to \infty} \frac{4}{3x^3}$

	$\displaystyle = \frac{4}{3}\lim_{x \to \infty} \frac{1}{3x^3}$

	$\displaystyle = \frac{4}{3}*\frac{1}{\infty}$

	$ = 0$

\end{enumerate}	



	



\fi
\newpage


%%%%%%%%%%%%%%%%%%%%%%%%%%%%%%%%%%%%%%%%%%%%%%%%%%%%%%%%
% PROBLEM THREE %% PROBLEM THREE %% PROBLEM THREE %% PROBLEM THREE %% PROBLEM THREE %
%==============================================================================
% Problem 3: Limits at Infinity Review
%==============================================================================
% PROBLEM THREE %% PROBLEM THREE %% PROBLEM THREE %% PROBLEM THREE %% PROBLEM THREE %
%%%%%%%%%%%%%%%%%%%%%%%%%%%%%%%%%%%%%%%%%%%%%%%%%%%%%%%%

\vspace{5mm}

\item Compute the following limits at infinity.  Show all work and justify your answer.

\begin{enumerate}
\item For real numbers $m, n > 0$ compute $ \displaystyle \lim_{x \to \infty} \frac{x^m}{e^{nx}} $
\item What does this tell us about the rate at which $e^{nx}$ approaches infinity relative to $x^m$? A brief explanation is fine for this part.
\item For real numbers $m, n > 0$ compute $ \displaystyle  \lim_{x \to  \infty} \frac{(\ln  x)^n}{x^m} $
\item What does this tell us about the rate at which $(\ln  x)^n$ approaches infinity relative to $x^m$? A brief explanation is fine for this part.
\end{enumerate}
\if\solutions1
\vspace{2mm}

\textbf{Solution:} \\
%==============================================================================
% STUDENTS: TYPE YOUR SOLUTIONS HERE. (Between \textbf{Solution:} and \fi )
%==============================================================================
\begin{enumerate}
	\begin{multicols}{2}
		\item $ \displaystyle \lim_{x \to \infty} \frac{x^m}{e^{nx}} $
		
		$\displaystyle =\lim_{x \to \infty} \frac{x^{m-1}* m}{e^{nx}*n}$ L'hopital
		
		$\displaystyle =\lim_{x \to \infty} \frac{x^{m-2}* m * (m-1)}{e^{nx}*n^2}$ L'hopital

		The numerator eventually eliminates the variable $x$ while the denominator always approaches $\infty$. 
		
		$\therefore \displaystyle \lim_{x \to \infty} \frac{x^m}{e^{nx}} = \frac{m}{\infty} = 0 $

		\item The limit being 0 tells us the denominator $(e^{nx})$ grows faster than  the numerartor $(x^m)$, that is, it tells us that $e^{nx}$ approaches infinity much faster relative to $x^m$.
		\vskip 0.9in
		\phantom{.}
	
	\end{multicols}
	\vspace{1cm}
	\begin{multicols}{2}
		\item $ \displaystyle  \lim_{x \to  \infty} \frac{(\ln  x)^n}{x^m} $
		
		$ \displaystyle  =\lim_{x \to  \infty} \frac{\frac{n\ln ^{n-1}(x)}{x}}{mx^{m-1}} $ L'hopital

		$ \displaystyle  =\lim_{x \to  \infty} \frac{n\ln ^{n-1}(x)}{mx^{m-1}} $

		The numerator eventually becomes a constant $m$ (eliminating the variable $x$) while the denominator always approaches $\infty$. 

		$\therefore \displaystyle \lim_{x \to \infty} \frac{(\ln  x)^n}{x^m} = \frac{m}{\infty} = 0 $

		\item The limit being 0 tells us the denominator $(x^m)$ grows faster than  the numerartor $(\ln  x)^n$, that is, it tells us that $(x^m)$ approaches infinity much faster relative to $(\ln  x)^n$.
		\vskip 0.9in
		\phantom{.}

	\end{multicols}

\end{enumerate}	


\fi

\newpage

%==============================================================================
% Problem 4: Root and Ratio Test Review
%==============================================================================
\vspace{5mm}

\item The problems in this question deal with the Root and the Ratio Tests. Determine the convergence or divergence of the following series. State which test you used.

\begin{enumerate}
\item  $ \displaystyle \sum_{n=1}^{\infty} \frac{e^{2n}}{n^n} $
\item $ \displaystyle \sum_{n=0}^{\infty} \frac{2^n}{n!} $
\item $ \displaystyle \sum_{n=0}^{\infty} \frac{n^2  2^{n+1}}{3^n}  $
\item $ \displaystyle \sum_{n=1}^{\infty} \left ( \frac{\ln n}{n} \right )^n $
\end{enumerate}

\if\solutions1
\vspace{3mm}
{\bf Solution}: \\
%==============================================================================
% STUDENTS: TYPE YOUR SOLUTIONS HERE. (Between \textbf{Solution:} and \fi )
%==============================================================================
\begin{enumerate}
	\begin{multicols}{2}

		\item  $ \displaystyle \sum_{n=1}^{\infty} \frac{e^{2n}}{n^n} $
		
		$ \displaystyle L = \lim_{n \to \infty} \sqrt[n]{\left|\frac{e^{2n}}{n^n}\right|} $
		$ \displaystyle =\lim_{n \to \infty} \frac{e^{2}}{n} $
		$=0$

		Seeing that $L = 0$, by the root test if $L < 1$, the series convergers, thus this series is absolutely convergent.

		\phantom{.}

		\item $ \displaystyle \sum_{n=0}^{\infty} \frac{2^n}{n!} $
		
		Let $a_n = \frac{2^n}{n!}$ hence $a_{n+1} = \frac{2^{n+1}}{(n+1)!}$
		
		$ \displaystyle L = \lim_{n \to \infty} \left(\frac{2^{n+1}}{(n+1)!}*\frac{n!}{2^n}\right) = \lim_{n \to \infty}\frac{2*n!}{(n+1)*n!}$
		$=\lim_{n \to \infty} \frac{2}{(n+1)} = 0$

		$L=0<1$

		By the ratio test, since $L < 1 $ this series is absolutely convergent.

	\end{multicols}

	\vskip 0.5cm

	\begin{multicols}{2}

		\item $ \displaystyle \sum_{n=0}^{\infty} \frac{n^2  2^{n+1}}{3^n}  $
		
		Let $a_n = \frac{n^2  2^{n+1}}{3^n}$ hence $a_{n+1} = \frac{(n+1)^2  2^{n+2}}{3^{n+1}}$
		
		$ \displaystyle L = \lim_{n \to \infty} \frac{(n+1)^2 * 2^n *4}{3^n * 3} * \frac{3^n}{n^2 * 2^{n} * n}$

		$ \displaystyle = \lim_{n \to \infty} \frac{6n + 2}{3} = \infty$

		$ \infty > 1$

		By the ratio test, since $L > 1 $ this series is divergent.

		\item $ \displaystyle \sum_{n=1}^{\infty} \left ( \frac{\ln n}{n} \right )^n $
		
		$ \displaystyle L = \lim_{n \to \infty} \sqrt[n]{\left | \frac{\ln n}{n} \right |^n } = \lim_{n \to \infty}\frac{\ln n}{n} $

		$=\lim_{n \to \infty}\left(\frac{\frac{1}{n}}{1}\right)$ L'hopital 
		
		$= \frac{1}{\infty} = 0 $

		By the root test $L < 1$, therefore this series is absolutely convergent.
		

	\end{multicols}

\end{enumerate}


\fi


%========================================================================================================================

\een 


\end{document}
